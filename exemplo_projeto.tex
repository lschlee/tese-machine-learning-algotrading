%%
%% This is file `example.tex',
%% generated with the docstrip utility.
%%
%% The original source files were:
%%
%% coppe.dtx  (with options: `example')
%% 
%% This is a sample monograph which illustrates the use of `coppe' document
%% class and `coppe-unsrt' BibTeX style.
%% 
%% \CheckSum{1416}
%% \CharacterTable
%%  {Upper-case    \A\B\C\D\E\F\G\H\I\J\K\L\M\N\O\P\Q\R\S\T\U\V\W\X\Y\Z
%%   Lower-case    \a\b\c\d\e\f\g\h\i\j\k\l\m\n\o\p\q\r\s\t\u\v\w\x\y\z
%%   Digits        \0\1\2\3\4\5\6\7\8\9
%%   Exclamation   \!     Double quote  \"     Hash (number) \#
%%   Dollar        \$     Percent       \%     Ampersand     \&
%%   Acute accent  \'     Left paren    \(     Right paren   \)
%%   Asterisk      \*     Plus          \+     Comma         \,
%%   Minus         \-     Point         \.     Solidus       \/
%%   Colon         \:     Semicolon     \;     Less than     \<
%%   Equals        \=     Greater than  \>     Question mark \?
%%   Commercial at \@     Left bracket  \[     Backslash     \\
%%   Right bracket \]     Circumflex    \^     Underscore    \_
%%   Grave accent  \`     Left brace    \{     Vertical bar  \|
%%   Right brace   \}     Tilde         \~}
%%
\documentclass[grad,numbers]{coppe}
\usepackage{amsmath,amssymb}
\usepackage{hyperref}
\usepackage[utf8]{inputenc}
\usepackage[brazil]{babel}
\usepackage[T1]{fontenc}
\usepackage{graphicx}

\makelosymbols
\makeloabbreviations

\begin{document}
  \title{Título da Tese}
  \foreigntitle{Thesis Title}
  \author{Lucas}{Schlee de Brito Fernandes}
  \advisor{Prof.}{Nome do Primeiro Orientador}{Sobrenome}{D.Sc.}
  \advisor{Prof.}{Nome do Segundo Orientador}{Sobrenome}{Ph.D.}
  \advisor{Prof.}{Nome do Terceiro Orientador}{Sobrenome}{D.Sc.}

  \examiner{Prof.}{Nome do Primeiro Examinador Sobrenome}{D.Sc.}
  \examiner{Prof.}{Nome do Segundo Examinador Sobrenome}{Ph.D.}
  \examiner{Prof.}{Nome do Terceiro Examinador Sobrenome}{D.Sc.}
  \examiner{Prof.}{Nome do Quarto Examinador Sobrenome}{Ph.D.}
  \examiner{Prof.}{Nome do Quinto Examinador Sobrenome}{Ph.D.}
  
  
  
  \department{ECA}% Confira a tabela a seguir para saber como preencher o comando \department de acordo com seu curso (Graduação - Poli) ou programa (Pós-Graduação - COPPE).
  
  %%%%%% Para alunos da POLI %%%%%%
  
  %% Course											Option
  %% Engenharia Ambiental                             EA
  %% Engenharia Civil                                 ECV
  %% Engenharia de Computação e Informação            ECI
  %% Engenharia de Controle e Automação               ECA
  %% Engenharia de Materiais                          EMAT
  %% Engenharia de Petróleo                           EPT
  %% Engenharia de Produção                           EPR
  %% Engenharia Eletrônica e de Computação            EEC
  %% Engenharia Elétrica                              EET
  %% Engenharia Mecânica                              EMC
  %% Engenharia Metalúrgica                           EMET
  %% Engenharia Naval e Oceânica                      ENO
  %% Engenharia Nuclear                               ENU
  
  
  %%%%%% Para alunos da COPPE %%%%%%
  
  %% Program											Option
  %% Engenharia Biomédica								PEB
  %% Engenharia Civil									PEC
  %% Engenharia Elétrica								PEE
  %% Engenharia Mecânica								PEM
  %% Engenharia Metalúrgica e de Materiais				PEMM
  %% Engenharia Nuclear									PEN
  %% Engenharia Oceânica								PENO
  %% Planejamento Energético							PPE
  %% Engenharia de Produção								PEP
  %% Engenharia Química									PEQ
  %% Engenharia de Sistemas e Computação				PESC
  %% Engenharia de Transportes							PET
  
  
  
  
  
  
  \date{01}{2016}

  \keyword{Primeira palavra-chave}
  \keyword{Segunda palavra-chave}
  \keyword{Terceira palavra-chave}

  \maketitle

  \frontmatter
  
  \makecatalog
  
  \dedication{A alguém cujo valor é digno desta dedicatória.}

  \chapter*{Agradecimentos}

  Gostaria de agradecer a todos.

  \begin{abstract}

  Apresenta-se, nesta tese, ...

  \end{abstract}

  \begin{foreignabstract}

  In this work, we present ...

  \end{foreignabstract}

  \tableofcontents
  \listoffigures
  \listoftables
  \printlosymbols
  \printloabbreviations

  \mainmatter
%  \doublespacing
  \chapter{Introdução}
  
  \section{Tema}
    
    \paragraph{}O projeto consiste na aplicação de inteligência articificial às negociações de alta frequência no mercado financeiro (High Frequency Trading). Nesse contexto, o problema a ser resolvido é o levantamento de padrões de mercado que indiquem uma previsão futura, a curto prazo, de uma variação positiva no preço de um determinado ativo.
    
  \section{Delimitação}

    \paragraph{}Toda a base de dados construída para o presente trabalho foi proveniente dos dados reais negociados na Bolsa de Valores de São Paulo (BM\&FBOVESPA), que são de domínio público. 
    
    \paragraph{}Por conta da granularidade dos dados obtidos pela fonte acima, um posterior trabalho de pré-processamento precisou ser feito para que os dados fossem agrupados em pequenos intervalos de tempo e, dessa forma, pudessem ter maior valor para o modelo.
  
  \section{Justificativa}
  
    \paragraph{}Algoritmos de negociação vêm ganhando cada vez mais força conforme a globalização se intensifica. Nos Estados Unidos, estima-se que aproximadamente ${65\%}$ de todo o volume negociado na bolsa de valores seja movimentado por agentes autônomos (robôs) executando algoritmos de HFT (High Frequency Trading).  
  
    \paragraph{}Apesar do conceito de HFT já estar difundido e vastamente aplicado, grande parte das estratégias desse grupo de algoritmos baseia-se em análises fundamentalistas de mercado, exigindo um conhecimento aprofundado do domínio e um refinamento milimetricamente calibrado na implementação de regras que irão executar a negociação. 
    
    \paragraph{}Ainda assim, um algoritmo que executa puramente um conjunto de regras não é totalmente seguro para realizar negociações. Dessa forma, a aplicação de técnicas de aprendizado de máquina podem auxiliar no poder preditivo desses algoritmos, identificando padrões passados e gerando sinais para tomadas de decisão mais seguras.
    
  \section{Objetivos}
  
    \paragraph{}O objetivo do trabalho é desenvolver e modelar estratégias direcionais de negociação de alta frequência, ou seja, indicar se um determinado ativo apresentará alta no período imediatamente após o intervalo em que foi analisado. Sendo assim, se o modelo indicar que haverá alta, será enviada uma ordem de compra à mercado e uma ordem de venda à mercado assim que o preço do ativo sendo avaliado ultrapassar um limiar de preço ou intervalo de tempo pré-determinado. 
  
  \section{Metodologia}
  
    \paragraph{}Com o intuito de otimizar o pré-processamento dos dados brutos da bolsa de valores, inicialmente agrupados em \textit{ticks}, foi implementada uma solução em \textit{C\#} utilizando o \textit{framework} multiplataforma \textit{.NET Core}, podendo ser executado em todos os sistemas operacionais. Por ser uma linguagem compilada e performática, obteve-se um bom desempenho na descompactação dos arquivos comprimidos contendo as negociações, no processamento dos dados e na exclusão posterior do arquivo descompactado para economizar memória em disco.
    
    \paragraph{}A principal função dessa etapa de pré-processamento dos dados brutos é o agrupamento em janelas de tempo para que indicadores de análise técnica, geralmente aplicados em intervalos diários, pudessem ser aplicados em intervalos de minutos. Dessa forma, o intervalo pode conter informações históricas a partir desses indicadores, gerando, por exemplo, valores de médias móveis ou indicadores que dependam de intervalos anteriores. 
    
    \paragraph{}A saída do modelo, \textbf{y}, pode assumir valores de 1 e 0, indicando que deve ou não ser enviada uma ordem de compra ou, mais precisamente, que haverá alta ou baixa no preço do ativo no próximo minuto, respectivamente.
    
    \paragraph{}Após a etapa de agrupamento dos dados e geração de indicadores, os dados são enviados para um arquivo \textit{CSV (Comma-Separated Values)} para serem consumidos pelo modelo.
    
    \paragraph{}Na etapa de consumo dos dados gerados, foi utilizada a linguagem de programação \textit{Python} e algumas bibliotecas orientadas para a ciência de dados, como o \textit{scikit-learn}, \textit{Pandas}, \textit{Numpy}, \textit{Seaborn} e \textit{MatplotLib}.
    
    \paragraph{}Com o auxílio das diversas opções de normalização dos dados oferecidas pelo \textit{scikit-learn}, foram feitas normalizações logarítmicas e de máximo e mínimo para as \textit{features} o conjunto de dados. Dessa forma, todas as variáveis de entrada do modelo possuíam variações entre 0 e 1.
    
    \paragraph{}Através do \textit{scikit-learn}, também foram utilizados modelos de aprendizado de máquina supervisionado para a previsão dos dados de teste. Dessa maneira, o MLP (\textit{MultiLayer Perceptron}) e a Regressão Logística foram escolhidos como modelos para a previsão dos dados.
    
    \paragraph{}Por fim, para a validação dos modelos obtidos, foi utilizada uma variação da validação cruzada para séries temporais, de forma que, para um determinado intervalo de tempo, os dados de testes fossem sempre futuros em relação aos dados de treino.
  
  \section{Descrição}
  
    \paragraph{}No capítulo 2 são levantados alguns fundamentos teóricos sobre HFT, assim como a contextualização sobre a "leitura de fita", base para entender o conjunto bruto de dados e realizar seu futuro pré-processamento.
    
    \paragraph{}O capítulo 3 introduz as informações provenientes dos dados brutos da BM\&FBOVESPA, assim como as técnicas utilizadas para pré-processá-los. Nele, também serão discutidos os diversos indicadores de análise técnicas aproveitados como \textit{features} no modelo e como foi realizada a normalização desses indicadores.
    
    \paragraph{}No capítulo 4, são discutidos os modelos utilizados para a previsão dos dados de teste e as técnicas de validação utilizadas.
    
    \paragraph{}O capítulo 5 mostra os resultados e discussões obtidos para um determinado ativo do mercado brasileiro.
    
    \paragraph{}Por fim, no capítulo 6 são levantadas as conclusões do projeto, indicando as limitações encontradas e sugestões para evoluções do trabalho realizado.
  
  \section{Exemplo}
    Segundo a norma de formatação de teses e dissertações do
  Instituto Alberto Luiz Coimbra de Pós-graduação e Pesquisa de
  Engenharia (COPPE), toda abreviatura deve ser definida antes de
  utilizada.\abbrev{COPPE}{Instituto Alberto Luiz Coimbra de Pós-graduação e Pesquisa de Engenharia}
  
    Do mesmo modo, é imprescindível definir os símbolos, tal como o
  conjunto dos números reais $\mathbb{R}$ e o conjunto vazio $\emptyset$.
  \symbl{$\mathbb{R}$}{Conjunto dos números reais}
  \symbl{$\emptyset$}{Conjunto vazio}
    
    Você deve selecionar seu curso de engenharia usando o comando \texttt{\textbackslash department\{Sigla\}} e no lugar de Sigla inserir a sigla referente ao seu curso de engenharia. A tabela \ref{tab:courses} relaciona as siglas dos cursos de engenharia da Escola Politécnica da Universidade Federal do Rio de Janeiro (POLI-UFRJ), enquanto que a tabela \ref{tab:programs} relaciona as siglas dos programas de pós graduação da COPPE.\abbrev{POLI-UFRJ}{Escola Politécnica da Universidade Federal do Rio de Janeiro}
  
  \begin{table}[h]
    \caption{Siglas dos cursos de engenharia da Escola Politécnica da UFRJ.}
    \label{tab:courses}
    \centering
    {\footnotesize
    \begin{tabular}{|c|c|}
      \hline
      Sigla & Curso\\
      \hline
      EA &  Engenharia Ambiental \\
      ECV & Engenharia Civil\\
      ECI & Engenharia de Computação e Informação \\
      ECA & Engenharia de Controle e Automação \\
      EMAT & Engenharia de Materiais\\
      EPT & Engenharia de Petróleo\\
      EPR & Engenharia de Produção\\
      EEC & Engenharia Eletrônica e de Computação\\
      EET & Engenharia Elétrica\\
      EMC & Engenharia Mecânica\\
      EMET & Engenharia Metalúrgica\\
      ENO & Engenharia Naval e Oceânica\\
      ENU & Engenharia Nuclear\\
      \hline
    \end{tabular}}
    \end{table}
    
    
  \begin{table}[h]
  	\caption{Siglas dos programas de pós graduação da COPPE.}
  	\label{tab:programs}
  	\centering
  	{\footnotesize
  	\begin{tabular}{|c|c|}
  		\hline
  		Sigla & Curso\\
  		\hline
  		PEB & Engenharia Biomédica \\
  		PEC & Engenharia Civil\\
  		PEE & Engenharia Elétrica \\
  		PEM & Engenharia Mecânica \\
  		PEMM & Engenharia Metalúrgica e de Materiais\\
  		PEN & Engenharia Nuclear\\
  		PENO & Engenharia Oceânica\\
  		PPE & Planejamento Energético\\
  		PEP & Engenharia de Produção\\
  		PEQ & Engenharia Química\\
  		PESC & Engenharia de Sistemas e Computação\\
  		PET & Engenharia de Transportes\\
  		\hline
  	\end{tabular}}
  \end{table}


  Note também que todas as figuras ou tabelas devem ser citadas no texto. Como ocorre com as tabelas \ref{tab:courses} e \ref{tab:programs}. Para ilustrar o uso de figuras em \LaTeX, considere as figuras \ref{fig:poli} e \ref{fig:coppe}.
  
   \begin{figure}
      \centering
      \includegraphics[width=5cm]{poli-logo.pdf}
      \caption{Logotipo da POLI-UFRJ.}
      \label{fig:poli}
    \end{figure}
    
    \begin{figure}
       \centering
       \includegraphics[width=5cm]{coppe-logo.pdf}
       \caption{Logotipo da COPPE.}
       \label{fig:coppe}
     \end{figure}

  \chapter{Revisão Bibliográfica}

  Para ilustrar a completa adesão ao estilo de citações e listagem de
  referências bibliográficas, a Tabela \ref{tab:citation} apresenta citações de alguns dos trabalhos contidos na norma fornecida pela CPGP da
  COPPE, utilizando o estilo numérico.

  \begin{table}[h]
  \caption{Exemplos de citações utilizando o comando padrão
    \texttt{\textbackslash cite} do \LaTeX\ e
    o comando \texttt{\textbackslash citet},
    fornecido pelo pacote \texttt{natbib}.}
  \label{tab:citation}
  \centering
  {\footnotesize
  \begin{tabular}{|c|c|c|}
    \hline
    Tipo da Publicação & \verb|\cite| & \verb|\citet|\\
    \hline
    Livro & \cite{book-example} & \citet{book-example}\\
    Artigo & \cite{article-example} & \citet{article-example}\\
    Relatório & \cite{techreport-example} & \citet{techreport-example}\\
    Relatório & \cite{techreport-exampleIn} & \citet{techreport-exampleIn}\\
    Anais de Congresso & \cite{inproceedings-example} &
      \citet{inproceedings-example}\\
    Séries & \cite{incollection-example} & \citet{incollection-example}\\
    Em Livro & \cite{inbook-example} & \citet{inbook-example}\\
    Dissertação de mestrado & \cite{mastersthesis-example} &
      \citet{mastersthesis-example}\\
    Tese de doutorado & \cite{phdthesis-example} & \citet{phdthesis-example}\\
    \hline
  \end{tabular}}
  \end{table}
  
  É importante notar que, segundo a \href{http://www.poli.ufrj.br/graduacao_projeto.php}{Norma para a Elaboração Gráfica do Projeto de Graduação} da Escola Politécnica da UFRJ para trabalhos de conclusão de curso de engenharia de julho de 2012, as referências bibliográficas podem ser apresentadas de duas formas: $(i)$ Referências numeradas e $(ii)$ Referências em ordem alfabética. Para exibição numerada, em que a exibição das referências bibliográficas segue a ordem de citação usada no texto, use o comando \texttt{\textbackslash bibliographystyle\{coppe-unsrt\}}. Para exibição de referências bibliográficas em ordem alfabética, basta usar o comando \texttt{\textbackslash bibliographystyle\{coppe-plain\}} ao final do documento. 
  
  
  \chapter{Método Proposto}
  
  
  
  \chapter{Resultados e Discussões}
  
  
  
  \chapter{Conclusões}
  
  
  
  

  \backmatter
  \bibliographystyle{coppe-unsrt}
  \bibliography{example}

  \appendix
  \chapter{Algumas Demonstrações}
\end{document}
%% 
%%
%% End of file `example.tex'.
